\documentclass[12pt,a4paper]{article}
\usepackage{geometry}
\usepackage{slashbox}
\geometry{
	a4paper,
	total={170mm,257mm},
	left=20mm,
	right=20mm,
	top=20mm,
	bottom=20mm
}
\usepackage{graphicx}
\usepackage{pdfpages}
\usepackage{placeins}
\usepackage{float}

\usepackage{polski}
\usepackage[utf8]{inputenc}

\begin{document}
	
	\begin{titlepage}
		\newgeometry{top=5.5cm, bottom=3cm}
		
		\centering
		{\huge\bfseries Logika układów cyfrowych lab.\par}
		
		\vspace{0.5cm}
		Prowadzący: Mgr inż. Antoni Sterna (E02-38m, wtorek 17:05) \\
	
		\vspace{1.1cm}
		{\Large sprawozdanie 4 - 2017.11.07\par}
		\vfill
		
		{\large\bfseries Jakub Dorda 235013\par}
		{\large\bfseries Marcin Kotas 235098\par}
		
		\vspace{1cm}
		\today \\ \LaTeX
		
		\restoregeometry
	\end{titlepage}

	\newgeometry{top=1.5cm, bottom=1.5cm, left=20mm, right=20mm}

	\section{Wprowadzenie/cel ćwiczeń}
		
	\section{Subtraktor szeregowy}
	
		\subsection{Grafy:}
	
			\begin{center}
				\makebox[\textwidth]{\includegraphics[width=\paperwidth - 90mm]{schem/diag1.png}}
				Graf 1 - subtraktor szeregowy w wersji Moore'a
			\end{center}
			
			
			\begin{center}
				\makebox[\textwidth]{\includegraphics[width=\paperwidth - 145mm]{schem/diag2.png}}
				Graf 2 - subtraktor szeregowy w wersji Mealy
			\end{center}
		
		\FloatBarrier
		\restoregeometry
		
	\section{Komparator szeregowy}
	
		\subsection{Grafy:}
		
			\begin{center}
				\makebox[\textwidth]{\includegraphics[width=\paperwidth - 30mm]{schem/diag3.png}}
				Graf 3 - komparator szeregowy w wersji Moore'a
			\end{center}
			
			\vspace{1.5cm}
			\begin{center}
				\makebox[\textwidth]{\includegraphics[width=\paperwidth - 30mm]{schem/diag4.png}}
				Graf 4 - komparator szeregowy w wersji Mealy
			\end{center}
	
		\subsection{Tabela prawdy i tablice Karnaugh:}
		
		
		\subsection{Minimalizacje:}
		
		
		\subsection{Użyte wzory:}
		
		
		\subsection{Schemat układu:}
		
		\vspace{1.5cm}
		\begin{center}
			\makebox[\textwidth]{\includegraphics[width=\paperwidth - 30mm]{schem/circuit.png}}
			Schemat 1 - komparator szeregowy w wersji Mealy
		\end{center}

	\section{Wnioski/podsumowanie}
	
			W celu sprawdzenia poprawności działania należało przeprowadzić testy dla wszystkich możliwych kombinacji wejść. Pierwsze ćwiczenie zostało wykonane poprawnie, natomiast drugie ćwiczenie zostało zaprojektowane bez pomocy tabeli prawdy, przez co uruchomienie układu się nie powiodło. Poprawny model układu został umieszczony w sprawozdaniu.
	
\end{document}